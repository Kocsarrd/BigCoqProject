\documentclass{article}
\usepackage[utf8]{inputenc}
\usepackage{amsmath}
\usepackage{amssymb}
\usepackage{graphicx}
\usepackage{marvosym}
\usepackage[a4paper, margin=2.5cm]{geometry}

\usepackage{listings}


\title{Programming languages with exceptions}
\author{Bálint Kocsis \\ e-mail: \texttt{balint.kocsis@ru.nl} \\ student number: 1084447
\and
Márk Széles \\ e-mail: \texttt{mark.szeles@ru.nl} \\ student number: 1082002}
\date{\today}

\begin{document}

\maketitle

\section{Introduction}

\subsection{Summary of our project}

The first goal of our project was to extend the language of \texttt{week9} with tagged exceptions. For this we fix a set \texttt{tag} of exception tags, and introduce the following constructs to our language:
\begin{lstlisting}
  EThrow : tag -> expr -> expr
  ECatch : expr -> tag -> expr -> expr
\end{lstlisting}
The intuitive semantics is that \texttt{EThrow} throws an exception annotated with a tag and \texttt{ECatch} can catch exceptions with a given tag and apply the handler to the thrown value. A few examples:

\begin{lstlisting}[mathescape]
  catch (throw MyException 0) MyException ($\lambda$ e, e)
\end{lstlisting}
Since the tags match, \texttt{catch} catches the exception, and returns \texttt{0}.

\begin{lstlisting}
  catch (throw MyException 0) MyOtherException ($\lambda$ e, e)
\end{lstlisting}
This expression yields and exception, as the the thrown exception tag does not match with the given tag in the \texttt{catch} expression.

We then present a big-step operational semantics for the language which is based on the big-step semantics of the \texttt{week9}, but also describes exceptional behaviour.

Next, a wp-calculus is created for the language. The definition of weakest precodition now also contains an exceptional postcondition \texttt{EPhi}.
\begin{lstlisting}
  Definition wp (e : expr) (Phi : val -> sepProp) 
    (EPhi : tag -> val -> sepProp) : sepProp
\end{lstlisting}
Intuitively, the \texttt{wp} predicate expresses that the program \texttt{e} either returns with a value for which \texttt{Phi} holds or throws a tagged exception for which \texttt{EPhi} holds. 

The \texttt{proofmode} file is extended, so we can use our rules with Iris. To gain some confidence in our definitions a few example programs are specified and verified in the file \texttt{programs.v}

Finally, a linear type system is introduced for the language which is used the prove the type soundness theorem: a well-typed program never gets stuck. INSERT THEOREM HERE, POSSIBLY MENTION MEMORY LEAKS.

\subsection{Project structure}

Our project relies on the basic definitions and theorems in \texttt{week8.v}. The exceptional language is defined in \texttt{language.v}. In the same file, we define \texttt{wp} and prove the \texttt{wp} rules for each language constructor.

\texttt{proofmode.v} sets up the Iris proofmode (with the recommended modifications in the project description). It also contains defined language constructs (let, sum, pair, let, linear load) and the derived \texttt{wp} rules for these constructs, and also some necessary lemmas about substitution. Finally, we also defined language notation in this file.

A few example programs are verified in \texttt{programs.v} using the separation logic for the language.

\texttt{affine\_semtype.v} contains the semantic typing system for the language and the proof of the type soundness theorem.

\section{Language definition and big-step semantics}

\subsection{Syntax}

The syntax for our language is the same as the one in \texttt{week9}, extended with constructors for \texttt{throw} and \texttt{catch}.
\begin{lstlisting}
  EThrow : tag -> expr -> expr
  ECatch : expr -> tag -> expr -> expr
\end{lstlisting}
We did not include syntactic constructs for linear load, the let-pair construct, and constructors and eliminator for sum types.
These are encoded in the same way as we have seen in \texttt{week9} and \texttt{week11}.

\subsection{Big-step operational semantics}

The \texttt{subst} and \texttt{eval\_bin\_op} functions are defined as usual. 

Since programs now may result in an exception, the type of \texttt{big\_step} had to be modified. We introduced a type \texttt{result} representing the possible results of a program: it either terminates with a value or a tagged expression.
\begin{lstlisting}
  Inductive res :=
  | ok : val -> res
  | ex : tag -> val -> res.
\end{lstlisting}
The type of \texttt{big\_step} now becomes
\begin{lstlisting}
  Inductive big_step : expr -> heap -> res -> heap -> Prop
\end{lstlisting}
The rules for the existing constructs in \texttt{week9} remained the same, modulo rewriting the \texttt{val}s to \texttt{res}-s. New rules also had to be introduced to propagate exceptions. These rules are straightforward to derive. As an example here are the exceptional rules for operators:
\begin{lstlisting}
  | Op_big_step_ex_l h1 h2 e1 e2 op t v1 :
      big_step e1 h1 (ex t v1) h2 ->
      big_step (EOp op e1 e2) h1 (ex t v1) h2
  | Op_big_step_ex_r h1 h2 h3 e1 e2 op t v1 v2 :
      big_step e1 h1 (ok v1) h2 ->
      big_step e2 h2 (ex t v2) h3 ->
      big_step (EOp op e1 e2) h1 (ex t v2) h3
\end{lstlisting}

The rule for throw is also intuitive: if the expression parameter terminates in a value, then that value is thrown.
\begin{lstlisting}
  | Throw_big_step h1 h2 t e v :
      big_step e h1 (ok v) h2 ->
      big_step (EThrow t e) h1 (ex t v) h2
\end{lstlisting}
For catch there are three rules. One for the case when the guarded program terminates with a value:
\begin{lstlisting}
  | Catch_big_step h1 h2 e1 e2 t v1 :
       big_step e1 h1 (ok v1) h2 ->
       big_step (ECatch e1 t e2) h1 (ok v1) h2
\end{lstlisting}
There are two cases when the guarded program throws an exception, as \texttt{catch}  has to check for equality of tags:
\begin{lstlisting}
  | Catch_big_step_ex_1 h1 h2 h3 e1 e2 t v1 r :
      big_step e1 h1 (ex t v1) h2 ->
      big_step (EApp e2 (EVal v1)) h2 r h3 ->
      big_step (ECatch e1 t e2) h1 r h3
  | Catch_big_step_ex_2 h1 h2 e1 e2 t t' v1 :
      big_step e1 h1 (ex t' v1) h2 ->
      t <> t' ->
      big_step (ECatch e1 t e2) h1 (ex t' v1) h2.
\end{lstlisting}

Note that the second rule describes both when \texttt{e2} returns with a value and when it throws an exception. A similar compactness is achieved in other cases (e.g. \texttt{If\_true\_big\_step}). Had we chosen to describe the big-step operational semantics with mutually inductive predicates \texttt{big\_step} and \texttt{big\_step\_error}, we would have ended up with more rules. 

A mutual inductive definition also makes it a bit more inconvenient to do induction on the operational semantics, as the programmer has to look out for well-foundedness when using induction hypotheses. (In this project, this would not have been a big problem, but it is worth considering when designing the operational semantics.)

\section{Separation logic}

The definition of weakest precondition now involves an exceptional postcondition. Intuitively, \texttt{wp e Phi EPhi} holds if either \texttt{e} terminates with an \texttt{ok vret}, and the separation logic proposition \texttt{Phi} holds for \texttt{vret} or if \texttt{e} throws an exception \texttt{ex tag vexcept} and \texttt{EPhi} holds for \texttt{tag} and \texttt{vexcept}. This is made precise with the following definition:
\begin{lstlisting}
  Definition wp (e : expr) (Phi : val -> sepProp)
                           (EPhi : tag -> val -> sepProp) : sepProp := fun h =>
    forall hf, disjoint h hf ->
      exists r h',
        disjoint h' hf /\
        big_step e (union h hf) r (union h' hf) /\
        (Phi # EPhi) r h'.
\end{lstlisting}
where \texttt{\#} is notation for combining \texttt{Phi} and \texttt{EPhi} into a \texttt{res -> sepProp}.

Using this definition we proved the wp rules for all language constructs. In some cases it is known that a program always correctly terminates or always throws an exception. In these cases one of the postconditions can be arbitrary. For example the rules for values and \texttt{throw} are stated as
\begin{lstlisting}
  Lemma Val_wp Phi EPhi v :
    Phi v |~ wp (EVal v) Phi EPhi.
  
  Lemma Throw_wp Phi EPhi t e :
    wp e (EPhi t) EPhi |~ wp (EThrow t e) Phi EPhi.
\end{lstlisting}
Another interesting case is the rule for \texttt{catch} where we have to use the fact that equality of tags is decidable.
\begin{lstlisting}
  Lemma Catch_wp Phi EPhi e1 t e2 :
    wp e1 Phi (fun t' v1 => if tag_dec t t'
                              then wp (EApp e2 (EVal v1)) Phi EPhi
                              else EPhi t' v1) |~
    wp (ECatch e1 t e2) Phi EPhi.
\end{lstlisting}

The most complicated rule to prove is again the context rule. Recall that in \texttt{week9} an auxiliary lemma \texttt{big\_step\_ctx} was proven to decompose a statement of the form \texttt{big\_step (k e) h1 v3 h3} to two steps: first evaluating \texttt{e} to \texttt{v2} and then putting \texttt{v2} in the context, and evaluate it. For our exceptional language we also need a slightly modified version of this lemma to deal with goals and assumptions of the form \texttt{big\_step (k e) h1 (ok v3) h3}. 

Moreover, we also have to deal with cases when evaluating an expression \texttt{k e} results in an exception. This can happen in two cases: either \texttt{e} throws the exception or \texttt{e} evaluates to a value \texttt{v2}, and evaluating \texttt{k v2} raises the exception. This observation is made precise by the lemma \texttt{big\_step\_ctx\_ex}.
\begin{lstlisting}
  Lemma big_step_ctx_ex k e t v3 h1 h3 :
    ctx k ->
    big_step (k e) h1 (ex t v3) h3 <->
      big_step e h1 (ex t v3) h3 \/
        exists v2 h2,
         big_step e h1 (ok v2) h2 /\
         big_step (k (EVal v2)) h2 (ex t v3) h3.
\end{lstlisting}

\section{Semantic typing}

In this section we introduce a semantic typing system for the exceptional language and prove the type soundness theorem. It states that a well-typed program never gets stuck, it either returns with a value or throws an exception.
\begin{lstlisting}
  Definition terminates (e : expr) :Prop :=
    exists r h, big_step e NatMap.empty r h.
  
  Theorem type_soundness e A :
    typed [] e A ->
    terminates e.
\end{lstlisting}
Comparing this formulation to the type soundness theorem in \texttt{week9} there is a fundamental difference: we allow the program \texttt{e} to leak memory. Since well-typed programs may leak memory anyway, there is also no need to restrict \texttt{A} to be \texttt{copy}.

The presence of memory leaks is a consequence of the error propagation mechanism. We look at an example program to see the problem.
\begin{lstlisting}
  let x = alloc 0 in
  throw 42 + free x.
\end{lstlisting}
This program uses every variable exactly once, but still leaks memory, since after evaluating the left operand the expression \texttt{free x} will never evaluate. Control is given back to the caller, \texttt{x} goes out of scope, but its content remains on the heap.

This shows that a linear type system is too restrictive for the exceptional language. Instead we introduce an affine type system, which still prevents use-after-free bugs and is sufficient to prove the version of the type soundess theorem formulated at the beginning of this section.

To ease notation, we introduced the predicate transformer \texttt{abs} which allows an arbitrary disjoint heap next to the part on which the propositon holds.
\begin{lstlisting}
  Definition abs (P : sepProp) : sepProp := P ** TRUE.
\end{lstlisting}
For one and two parameter predicates, transformers \texttt{abs1} and \texttt{abs2} were defined in a similar manner.

Now, we define the semantics of types. Similar to \texttt{week11} we have
\begin{lstlisting}
  Definition ty := val -> sepProp.
\end{lstlisting}
To make our lives easier, we only allow expressions of type \texttt{nat} to be thrown as exceptions. Hence, we define the type of thrown exceptions to be
\begin{lstlisting}
  Definition TEx (t : tag) : ty := fun v => Ex n, @[ v = VNat n ].
\end{lstlisting}
The semantics of all but function types remains the same as in \texttt{week11}. For function types we allow some junk on the heap in the \texttt{wp} postconditions:
\begin{lstlisting}
  Definition TFun (A1 A2 : ty) : ty := fun v =>
    @[ forall w, A1 w |~ wp (EApp (EVal v) (EVal w)) (abs1 A2) (abs2 TEx) ].
  Definition TFunOnce (A1 A2 : ty) : ty := fun v =>
    All w, A1 w -** wp (EApp (EVal v) (EVal w)) (abs1 A2) (abs2 TEx).
\end{lstlisting}
The semantic typing judgments \texttt{val\_typed} and  \texttt{typed} also allow for memory leaks.
\begin{lstlisting}
  Definition val_typed (v : val) (A : ty) : Prop :=
    EMP |~ abs (A v).
  
  Definition typed (Gamma : ty_ctx) (e : expr) (A : ty) : Prop :=
    forall vs, ctx_typed Gamma vs |~ wp (subst_map vs e) (abs1 A) (abs2 TEx).
\end{lstlisting}
This is also necessary in case of \texttt{val\_typed} to be able to prove the typing rule of closures.

The definition of subtyping and subcontext also changes to account for memory leaks.
\begin{lstlisting}
  Definition subty (A1 A2 : ty) : Prop :=
    forall v, A1 v |~ abs (A2 v).
  
  Definition subctx (Gamma1 Gamma2 : ty_ctx) : Prop :=
    forall vs, ctx_typed Gamma1 vs |~ abs (ctx_typed Gamma2 vs).
\end{lstlisting}
Using these definitions it is possible the prove a general \texttt{subctx\_weakening} lemma (for non-copy types as well) and that every type is a subtype of \texttt{moved}.

Finally, type soundess follows from the \texttt{wp\_adequacy} theorem, which is similar to the one in \texttt{week11}, but contains the exceptional postcondition and allows for memory leaks.
\begin{lstlisting}
  Lemma wp_adequacy Phi EPhi e :
    (EMP |~ wp e Phi EPhi) ->
    exists r h, big_step e NatMap.empty r h /\ (Phi # EPhi) r h.
\end{lstlisting}

\section{Conclusion}

\section{Reflection}

\end{document}

