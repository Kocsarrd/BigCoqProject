\documentclass{article}
\usepackage[utf8]{inputenc}
\usepackage{amsmath}
\usepackage{amssymb}
\usepackage{graphicx}
\usepackage{marvosym}
\usepackage[a4paper, margin=2.5cm]{geometry}

\usepackage{listings}
\lstset{basicstyle=\small,
	  xleftmargin=6pt}

\title{A programming language with exceptions}
\author{Bálint Kocsis \\ e-mail: \texttt{balint.kocsis@ru.nl} \\ student number: 1084447
\and
Márk Széles \\ e-mail: \texttt{mark.szeles@ru.nl} \\ student number: 1082002}
\date{\today}

\begin{document}

\maketitle

\section{Introduction}

\subsection{Summary of our project}

The first goal of our project was to extend the language of \texttt{week9} with tagged exceptions. For this we fixed a set \texttt{tag} of
exception tags, and introduced the following constructs to our language:
\begin{lstlisting}
EThrow : tag -> expr -> expr
ECatch : expr -> tag -> expr -> expr
\end{lstlisting}
The intuitive semantics is that \texttt{EThrow} throws an exception annotated with a tag, and \texttt{ECatch} can catch exceptions with a
given tag and apply the handler to the thrown value. A few examples:
\begin{lstlisting}[mathescape]
catch (throw MyException 0) MyException ($\lambda$ e. e)
\end{lstlisting}
Since the tags match, \texttt{catch} catches the exception and returns \texttt{0}.
\begin{lstlisting}[mathescape]
let r := alloc 0 in
catch (throw MyException 0; r <- 1) MyOtherException ($\lambda$ e. e)
\end{lstlisting}
This expression yields and exception as the tag of the thrown exception does not match the given tag in the \texttt{catch} expression.
Also note that the assignment to \texttt{r} does not take place, so \texttt{r} still stores \texttt{0} after evaluation of the expression.

We added a big-step operational semantics to the language, which is based on the big-step semantics of \texttt{week9.v} but extended
to account for exceptional behaviour.

Next, a wp-calculus was created for the language, again based on the one in \texttt{week9.v}. The definition of weakest precondition now
also contains an exceptional postcondition \texttt{EPhi}.
\begin{lstlisting}
Definition wp (e : expr) (Phi : val -> sepProp)
                         (EPhi : tag -> val -> sepProp) : sepProp.
\end{lstlisting}
Intuitively, \texttt{wp e Phi EPhi} expresses that the program \texttt{e} either returns with a value for which \texttt{Phi} holds, or throws a
tagged exception for which \texttt{EPhi} holds.

To gain some confidence in our definitions, a few example programs have been specified and verified.

Finally, a linear type system similar to the one in \texttt{week11.v} has been introduced for the language which was used to the
prove a type soundness theorem: a well-typed program never gets stuck.

\subsection{Project structure}

Our project relies on the basic definitions and theorems of separation logic in \texttt{week8.v} (and hence, by transitivity, on \texttt{library.v}).
The language is defined in \texttt{language.v}. In the same file, we define \texttt{wp} and prove the quintessential rules of the wp-logic.

The file \texttt{proofmode.v} sets up the Iris proofmode (with the recommended modifications in the project description). It also contains some
helper lemmas and useful derived \texttt{wp} rules, as well as defined language constructs (let, sums, let-pair, linear load)
along with their derived \texttt{wp} rules. Additionally, nice notations for the embedded language are defined in this file.

A few example programs can be found in \texttt{programs.v}.  At the top of the file, the examples given in the project description
are checked against the big step semantics to confirm that they produce the expected vaue. After that, a recursive program \texttt{dec\_list} is
specified and verified using our customized wp-logic.

Finally, \texttt{affine\_semtype.v} contains the shallow embedding of a semantic typing system for the language and the proof of the type soundness theorem.

\section{Language definition and big-step semantics}

\subsection{Syntax}

The syntax for our language is the same as the one in \texttt{week9.v}, extended with constructors for \texttt{throw} and \texttt{catch}.
\begin{lstlisting}
EThrow : tag -> expr -> expr
ECatch : expr -> tag -> expr -> expr
\end{lstlisting}
We did not include let-expressions, sum types, the let-pair constructor and linear load in our syntax. Instead, these are encoded in the same way as we have
seen in \texttt{week9.v} and \texttt{week11.v} (see \texttt{proofmode.v}).

\subsection{Big-step operational semantics}

The \texttt{subst} and \texttt{eval\_bin\_op} functions are defined as usual. 

Since programs may now result in an exception, the type of \texttt{big\_step} had to be modified. We introduced a type \texttt{res} representing the possible
outcomes of a program: either an ordinary return value or a tagged exception.
\begin{lstlisting}
Inductive res :=
  | ok : val -> res
  | ex : tag -> val -> res.
\end{lstlisting}
The type of \texttt{big\_step} has thus become
\begin{lstlisting}
Inductive big_step : expr -> heap -> res -> heap -> Prop
\end{lstlisting}	

The rules for the existing constructs in \texttt{week9.v} remained the same, modulo replacing the \texttt{val}-s with \texttt{res}-s. However, new rules had to be
introduced to propagate exceptions. These rules are rather straightforward: if a subexpression throws an exception, then the whole expression
throws the same exception. As an example, here are the rules for operators:
\begin{lstlisting}
  | Op_big_step_ex_l h1 h2 op e1 e2 t v1 :
      big_step e1 h1 (ex t v1) h2 ->
      big_step (EOp op e1 e2) h1 (ex t v1) h2
  | Op_big_step_ex_r h1 h2 h3 op e1 e2 t v1 v2 :
      big_step e1 h1 (ok v1) h2 ->
      big_step e2 h2 (ex t v2) h3 ->
      big_step (EOp op e1 e2) h1 (ex t v2) h3
\end{lstlisting}
The rule for throw is also intuitive: if the expression parameter terminates in a value, then that value is thrown along with the given tag.
\begin{lstlisting}
  | Throw_big_step h1 h2 t e v :
      big_step e h1 (ok v) h2 ->
      big_step (EThrow t e) h1 (ex t v) h2
\end{lstlisting}
If the subexpression already throws an exception, then that exception is rethrown.

For catch there are three rules. The first is for the case when the guarded program (left argument) returns successfully:
\begin{lstlisting}
  | Catch_big_step h1 h2 e1 e2 t v1 :
      big_step e1 h1 (ok v1) h2 ->
      big_step (ECatch e1 t e2) h1 (ok v1) h2
\end{lstlisting}
The other two are for when the guarded program throws an exception. One rule is not enough as we need to check for equality of tags:
\begin{lstlisting}
  | Catch_big_step_ex_1 h1 h2 h3 e1 e2 t v1 r :
      big_step e1 h1 (ex t v1) h2 ->
      big_step (EApp e2 (EVal v1)) h2 r h3 ->
      big_step (ECatch e1 t e2) h1 r h3
  | Catch_big_step_ex_2 h1 h2 e1 e2 t t' v1 :
      big_step e1 h1 (ex t' v1) h2 ->
      t <> t' ->
      big_step (ECatch e1 t e2) h1 (ex t' v1) h2
\end{lstlisting}
Note that the rule \texttt{Catch\_big\_step\_ex\_1} describes both when \texttt{e2} returns successfully or with an exception. A similar compactness is achieved
in some other cases (e.g. \texttt{If\_true\_big\_step}).

An alternative solution we had thought of would be to define the big-step operational semantics in terms of two mutually inductive predicates \texttt{big\_step}
and \texttt{big\_step\_error} for ordinary and exceptional returns, respectively. Although this might lead to more convenient notation paper,
it would result in more rules for the operational semantics. For example, the rule \texttt{Catch\_big\_step\_ex\_1} would need to be split into two cases
depending on whether the handler throws an exception or not.

Furthermore, this approach would make it a bit more inconvenient to do induction on the operational semantics in Coq, as the programmer has to look out for
well-foundedness when using induction hypotheses. (In this project, this would not have been a big problem, but it is worth noting.)

\section{Separation logic}

The definition of weakest precondition now involves an exceptional postcondition. Intuitively, \texttt{wp e Phi EPhi} holds if either \texttt{e} terminates
successfully with \texttt{ok vret} and \texttt{Phi vret} holds, or if \texttt{e} throws an exception \texttt{ex t vex} and \texttt{EPhi t vex} holds.
This is made precise with the following definition:
\begin{lstlisting}
Definition wp (e : expr) (Phi : val -> sepProp)
                         (EPhi : tag -> val -> sepProp) : sepProp := fun h =>
  forall hf, disjoint h hf ->
    exists r h',
      disjoint h' hf /\
      big_step e (union h hf) r (union h' hf) /\
      (Phi # EPhi) r h'.
\end{lstlisting}
where \texttt{Phi \# EPhi} is notation for combining \texttt{Phi} and \texttt{EPhi} into a \texttt{res -> sepProp}.

Using this definition, we proved the wp rules for all language constructs. In some cases, it is known that a program always correctly terminates or
always throws an exception. In such situations, one of the postconditions can be arbitrary. For example, the rules for values and \texttt{throw} are stated as
\begin{lstlisting}
Lemma Val_wp Phi EPhi v :
  Phi v |- wp (EVal v) Phi EPhi.

Lemma Throw_wp Phi EPhi t e :
  wp e (EPhi t) EPhi |- wp (EThrow t e) Phi EPhi.
\end{lstlisting}
Another interesting case is the rule for \texttt{catch}, where we had to use the fact that equality of tags is decidable.
\begin{lstlisting}
Lemma Catch_wp Phi EPhi e1 t e2 :
  wp e1 Phi (fun t' v1 => if tag_dec t t'
                            then wp (EApp e2 (EVal v1)) Phi EPhi
                            else EPhi t' v1) |-
  wp (ECatch e1 t e2) Phi EPhi.
\end{lstlisting}

The most complicated rule to prove was the context rule. Recall that in \texttt{week9.v}, an auxiliary lemma \texttt{big\_step\_ctx} had been used to
decompose a statement of the form \texttt{big\_step (k e) h1 v3 h3} to two steps: first evaluating \texttt{e} to \texttt{v2}, then evaluting \texttt{k v2} to
\texttt{v3}. For our exceptional language, we need a slightly modified version of this lemma to deal with goals and assumptions of the form
\texttt{big\_step (k e) h1 (ok v3) h3}. 

Moreover, we also have to take into account the case when evaluating an expression \texttt{k e} results in an exception. This can happen in two ways:
either \texttt{e} throws the exception, or \texttt{e} evaluates to a value \texttt{v2} and evaluating \texttt{k v2} raises the exception. This observation is made
precise by the lemma \texttt{big\_step\_ctx\_ex}.
\begin{lstlisting}
Lemma big_step_ctx_ex k e t v3 h1 h3 :
  ctx k ->
  big_step (k e) h1 (ex t v3) h3 <->
    big_step e h1 (ex t v3) h3 \/
      exists v2 h2,
       big_step e h1 (ok v2) h2 /\
       big_step (k (EVal v2)) h2 (ex t v3) h3.
\end{lstlisting}

\section{Semantic typing}

In this section we present a semantic typing system for our language with exceptions and a proof of a type soundness theorem. It states that
a well-typed program never gets stuck: it either returns with a value or throws an exception.
\begin{lstlisting}
Definition terminates (e : expr) : Prop :=
  exists r h, big_step e NatMap.empty r h.

Theorem type_soundness e A :
  typed [] e A ->
  terminates e.
\end{lstlisting}
Comparing this formulation to the type soundness theorem in \texttt{week9.v} there is a fundamental difference: we allow the program \texttt{e} to leak
memory. Since well-typed programs may leak memory anyway, there is also no need to restrict \texttt{A} to be \texttt{copy}.

The presence of memory leaks is a consequence of the error propagation mechanism. Let us look at an example program to see the problem.
\begin{lstlisting}
let x = alloc 0 in
throw 42 + free x.
\end{lstlisting}
This program uses every variable exactly once, but still leaks memory: after evaluating the left operand of the addition, \texttt{free x} will never be executed.
Control is given back to the caller, causing \texttt{x} to go out of scope, with its content remaining on the heap without any references to it.

This shows that a linear type system is too restrictive for our language. Instead we used an affine type system, which still prevents use-after-free bugs, and
is sufficient to prove the version of the type soundess theorem formulated at the beginning of this section.

To ease notation and streamline our proofs, we defined the notion of an \textit{absorbing} proposition.
\begin{lstlisting}
Definition absorbing (P : sepProp) : Prop := P ** TRUE |- P.
\end{lstlisting}
Intuitively, \texttt{absoring P} expresses that we can freely give up any resources on the heap while proving \texttt{P}.

With this abstraction at hand, we are ready to prove some basic properties about absorbing propositions. For instance, we can define
the function \texttt{abs} which turns any proposition \texttt{P} to an absorbing proposition:
\begin{lstlisting}
Definition abs (P : sepProp) : sepProp := P ** TRUE.

Lemma abs_absorbing P :
  absorbing (abs P).
\end{lstlisting}
(Remark: \texttt{abs P} is the least specific absorbing proposition implied by \texttt{P}.)
For one and two parameter predicates, the transformers \texttt{abs1} and \texttt{abs2} were defined in a similar manner.

A crucial property of being absorbing is that it is preserved under weakest preconditions:
\begin{lstlisting}
Lemma wp_absorbing Phi EPhi e :
  (forall v, absorbing (Phi v)) ->
  (forall t v, absorbing (EPhi t v)) ->
  absorbing (wp e Phi EPhi).
\end{lstlisting}
This makes it possible to throw away resources when proving weakest preconditions. To make our proofs nicer, we defined a tactic \texttt{wp\_absorb}
which can absorb a set of spatial hypotheses when the goal is a \texttt{wp}. For an example of how these concepts are used in practice,
see the proof of one of the binary typing rules (e.g. \texttt{App\_typed}).

The actual semantics of types is similar to the one in \texttt{week11.v}. There are two essential differences:
\begin{itemize}
  \itemsep 0pt
  \item the presence of the exceptional postcondition in all \texttt{wp}-s
  \item making the postconditions absorbing, which in effect allows memory leaks to happen.
\end{itemize}

To make our lives easier, we only allow expressions of type \texttt{Nat} to be thrown as exceptions. Hence, we define the exceptional postcondition to be
\begin{lstlisting}
Definition TEx (t : tag) (v : val) : sepProp := Ex n, @[ v = VNat n ].
\end{lstlisting}
The semantics of all but function types remains the same as in \texttt{week11.v}. The semantics of function types needs to be changed mildly
to make the postconditions absorbing:
\begin{lstlisting}
Definition TFun (A1 A2 : ty) : ty := fun v =>
  @[ forall w, A1 w |- wp (EApp (EVal v) (EVal w)) (abs1 A2) (abs2 TEx) ].
Definition TFunOnce (A1 A2 : ty) : ty := fun v =>
  All w, A1 w -** wp (EApp (EVal v) (EVal w)) (abs1 A2) (abs2 TEx).
\end{lstlisting}
The situation is similar in the definition of the semantic typing judgment:
\begin{lstlisting}
Definition typed (Gamma : ty_ctx) (e : expr) (A : ty) : Prop :=
  forall vs, ctx_typed Gamma vs |- wp (subst_map vs e) (abs1 A) (abs2 TEx).
\end{lstlisting}

Analogously, the right hand side of some entailments were also needed to be made absorbing. As an example, this was the case for the definition
of subtyping (including subcontexts).
\begin{lstlisting}
Definition subty (A1 A2 : ty) : Prop :=
  forall v, A1 v |- abs (A2 v).

Definition subctx (Gamma1 Gamma2 : ty_ctx) : Prop :=
  forall vs, ctx_typed Gamma1 vs |- abs (ctx_typed Gamma2 vs).
\end{lstlisting}
Using these definitions made it possible the prove a general \texttt{subctx\_weakening} lemma (for non-copy types as well),
and that every type is a subtype of \texttt{Moved}.

Finally, type soundess follows from the \texttt{wp\_adequacy} theorem, which is similar to the one in \texttt{week11.v},
but contains the exceptional postcondition and allows for memory leaks.
\begin{lstlisting}
Lemma wp_adequacy Phi EPhi e :
  (EMP |- wp e Phi EPhi) ->
  exists r h, big_step e NatMap.empty r h /\ (Phi # EPhi) r h.
\end{lstlisting}

\section{Conclusion}

\section{Reflection}

\subsection{Márk Széles}

The main design decisions (regarding syntax for tags and exceptions, big-step semantics, the separation logic and the type system) about our project are the result of our discussions.

I implemented the definitions of the syntax and big-step operational semantics of the language. I proved about half of the wp rules in \texttt{language.v} and the wp rules for derived constructs in \texttt{proofmode.v}.

After that I started to work on sem	antic typing. First, I tried to implement a linear type system to be able to prove a stronger type soundness theorem (which also states the absence of memory leaks). Since that was not possible we settled on the affine type system presented in this document. Using the tactics created by my partner I proved most lemmas in \texttt{affine\_semtype.v} (except for a few typing rules and the type soundness theorem). 

Moreover, I wrote the first draft for the report which we then proofread and extended/modified together.

Overall, I would say the collaboration was good, we discussed many interesting ideas, especially about the semantic typing system.

\end{document}

